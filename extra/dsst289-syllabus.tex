\documentclass[11pt, a4paper]{article}

\usepackage{fontspec}
\usepackage{geometry}
\usepackage{fancyhdr}
\usepackage[hidelinks]{hyperref}
\usepackage[normalem]{ulem}
\usepackage{multicol}

\geometry{
  top=3cm,
  bottom=3cm,
  left=3cm,
  right=3cm,
  marginparsep=4pt,
  marginparwidth=1cm
}

\renewcommand{\headrulewidth}{0pt}
\pagestyle{fancyplain}
\fancyhf{}
\lfoot{}
\rfoot{}

\setlength{\parindent}{0pt}
\setlength{\parskip}{0pt}

\usepackage{xunicode}
\defaultfontfeatures{Mapping=tex-text}

\begin{document}

\begin{center}
\textbf{DSST289: Introduction to Data Science --- Taylor Arnold --- Spring 2024}
\end{center}

\vspace{0.5cm}

\textbf{Website}: \texttt{https://statsmaths.github.io/dsst289-s24}

\bigskip

\textbf{Topics:}
Methods for collecting, manipulating, visualizing, exploring, and presenting
data.

\bigskip

\textbf{Format:}
The semester is broken into three, four-week units. The first unit focuses on
data visualization and the second unit focuses on data collection. In the final
unit we will explore applications to spatial and temporal datasets. The
last week of the semester is dedicated to a final project. Class meetings are
hands-on and interactive. Please bring a computer, pencil, and something to
write on to each class. Class materials can be found on the course website.

\bigskip

\textbf{Homework:}
All class meetings, other than exam days, have a reading posted on our website.
A few questions or activities are included at the end of each reading. These
should be completed before class. You will self-reported completion through the class
form found at the top of the course website. We will complete these in class.
Please bring the written responses to class. Homework is graded based on the
percentage of assignments that were succesfully submitted.

\bigskip

\textbf{Exams:}
There are three midterm exams. Each has a take-home open-book component and 
a closed-book in-class component given on Wednesday. The take-home will be
distributed in advance of the in-class exam. Answers to the
take-home component should be printed out and handed in at the same time as
the in-class exam. A list of topics for the in-class exam will be posted on
the course website.

\bigskip

\textbf{Final Project:}
A final project will be due during the last week of class. The project will
focus on finding or creating a new dataset and applying the techniques learned
throughout the class to the analysis of it. The project will take the form of
a digital poster session and a one-page reflection. Detailed intructions will
be posted following fall break.

\bigskip

\textbf{Attendance:}
While regular attendance is expected, I understand that personal issues 
occasionally arise. Also, please do not come to class if you are feeling ill.
Excessive absenses (roughly, 5 or more) without a clear rational will result
in a warning followed by a possible reduction in your final grade. If you are
have a valid excuse to miss the in-class exams on the given date, it will be
replaced with an oral exam. The take-home portion should be submitted by email
at the appropriate time except in truly exceptional circumstances.

\bigskip

\textbf{Getting Help:}
We will usually have a lot of time in class to answer questions about the course
material. If questions arise outside of class, please also feel free to send
these by email. I am of course happy to schedule an office hours meeting for
any extended questions or personal concerns. Because I know everyone has a
busy schedule, I offer office hours primarily by appointment. Just send an
email with your availability at least 1 day before you'd like to meet. However,
note that I generally avoid answering conceptual questions about the homework
questions before they are due; just do your best and we will discuss in class.

\bigskip

\textbf{Final Grades:}
The three exam, final project, and homework grades are averaged (i.e., each
contributes 20\%). A letter grade is assigned as follows:
             A (93--100), A- (90--92),
B+ (87--89), B (83--86),  B- (80--82),
C+ (77--79), C (73--76),  C- (70--72), and F (0--69).

\end{document}
